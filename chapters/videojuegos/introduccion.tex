\par La industria de los videojuegos es uno de los pilares del entretenimiento actual. A nivel global, en 2024 se esperaba que la industria generara \$USD 187.7 miles de millones en ganancias \cite{buijsmanGlobalGamesMarket2024}. Argentina representa una parte muy pequeña de este mercado, y aún así en 2023 se estimó que el tamaño de la industria fue de \$USD 95.049.600 \cite{delaiglesiaObservatorioIndustriaArgentina2024}.
\bigbreak
\par Más allá de su valor económico, los videojuegos son una industria multidisciplinaria, que nuclea artistas, sonidistas, escritores, y lo más relevante para este trabajo, programadores e ingenieros de software.
\bigbreak
\par Este capítulo describe conceptos básicos de qué es un videojuego, además de adentrarse en la relación entre las metodologías del desarrollo y el desarrollo de uno. Se analizan frameworks utilizados, y sus posibles dificultades. También se habla sobre los ciclos de vida de un juego y el rol general de un ingeniero de software en su creación y gestión.
%
%
\section{Definición de videojuego}
\par Es importante definir qué es un videojuego. Según Scott Rogers \cite{rogersLevelGuiaPara2024}, un juego es una actividad que:
\begin{itemize}
    \item tiene al menos un jugador
    \item tiene reglas
    \item tiene una condición para perder o ganar
\end{itemize}
\par Un videojuego es, entonces, un juego que puede jugarse en una pantalla. Es un software interactivo, orientado principalmente al entretenimiento de las personas. Mediante distintos comandos y controles, se simula una experiencia en la pantalla de un dispositivo \cite{alarconaldanaMetodologiaParaDesarrollo2020}. 
Un videojuego es un producto sumamente complejo, ya que fusiona distintos aspectos del arte. Arte visual, sonido, narrativa y otros se utilizan para crear una experiencia interactiva \cite{garciaariasDesarrolloVideojuegosDesde2019}. Esta complejidad se verá reflejada a la hora de elegir una metodología de desarrollo.
\par A lo largo de esta sección, se utilizan palabras comunes al desarrollo del videojuego. Algunas de ellas son:
\begin{itemize}
    \item Mecánica: conjunto de reglas que definen cómo se juega un videojuego. Por ejemplo, en un juego de carreras, la mecánica puede ser que el jugador debe llegar a la meta antes que los demás competidores.
    \item Game designer: persona encargada de diseñar las mecánicas del videojuego. El game designer es responsable de definir cómo se juega, qué se puede hacer, y cómo se interactúa con el videojuego.
    \item Documento de diseño (GDD): documento que define las mecánicas del videojuego. También incluye la historia, los personajes, y otros aspectos relevantes del videojuego.
    \item Assets: un asset es un recurso que se utiliza en el videojuego. Puede ser un modelo 3D, una textura, un sonido, o cualquier otro recurso que se utilice en el videojuego.
    \item Playtesting: proceso en el cual se prueba un videojuego, generalmente por jugadores o testers, con el objetivo de identificar errores, evaluar la jugabilidad y obtener retroalimentación para mejorar la experiencia final de usuario.
\end{itemize} 

