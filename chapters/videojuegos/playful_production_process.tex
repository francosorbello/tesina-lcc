%
%  -- INTRODUCCION --
%
\par Richard Lemarchand es un game designer prominente en la industria de videojuegos. Trabajó en franquicias populares, como Gex, Soul Reaver y Uncharted \cite{RichardLemarchandcom}. Además, es profesor en la Universidad de Southern California (USC) \cite{RichardLemarchandcom}. En 2021, publicó un libro llamado “A Playful Production Process” \cite{lemarchandPlayfulProductionProcess2021}, donde explora una metodología de desarrollo basada en su experiencia como game designer, productor y profesor. \todo{TODO: Expandir} 
\par Para Lemarchand, hay una fuerte conexión entre game design y producción. Es por esto que en su metodología toma conceptos de ambos campos. De forma similar a Ramadan y Widyani, utiliza el libro "Game Design Workshop" \cite{fullertonGameDesignWorkshop2008} como punto de partida.
%
\begin{figure}[H]
  \centering
  \includegraphics[scale=0.3]{image6.png}
  \caption{Etapas de la metodología propuesta por Lemarchand. Extraída de \cite{lemarchandPlayfulProductionProcess2021}.}
  \label{fig:x proceso de desarrollo Lemarchand}
\end{figure}
%
%
%  -- IDEATION --
%
\subsection{Ideation}
El proceso de desarrollo comienza en estae etapa, donde se definen una serie de objetivos llamados \textit{player experience goals}. Este concepto es propuesto por Fullerton, que los define como ``una serie de objetivos propuestos por el game designer, que indican el tipo de experiencia que los jugadores tendran durante el videojuego." \cite{fullertonGameDesignWorkshop2008} (Traducción propia).
\par Ejemplos de estos objetivos son ``los jugadores sentirán curiosidad por explorar un universo desconocido.'' o ``los jugadores colaborarán para sobrevivir un mundo post apocalíptico.''. Para establecer estos objetivos, Lemarchand propone una serie de herramientas:
\begin{itemize}
    \item Brainstorming: activadad grupal o individual donde se generan ideas espontaneamente. Lemarchand recomienda enfocarse en la cantidad de ideas, sin preocuparse por su calidad, resaltando la importancia de escribir todas las ideas, incluso las que parecen malas o irrelevantes.
    \item Mind mapping: esta técnica consiste en escribir una idea central y luego conectar ideas relacionadas. 
    \begin{figure}[H]
        \centering
        \includegraphics[scale=0.3]{mind_map.png}
        \caption{Ejemplo de un \textit{mind map}. Extraída de \cite{lemarchandPlayfulProductionProcess2021}.}
        \label{fig:x ejemplo de un mind map Lemarchand}
    \end{figure} 
    \item Automastim: este ejercicio consiste en configurar un temporizador y escribir sin parar durante un tiempo determinado. El objetivo es generar ideas espontaneamente, sin preocuparse por la calidad de las mismas.
\end{itemize}
%
%
\subsubsection{Prototipado}
\par Realizar prototipos es una parte fundamental de esta etapa. Lemarchand recomienda empezar con a prototipar lo antes posible, enfocandose en explorar distintas ideas. El objetivo de un prototipo es encontrar una un número pequeño de mecánicas (inclusive una sola) que sea interesante o divertida. 
\par Para ayudar en la creación de prototipos, Lemarchand define algunos conceptos de game design. Primero define una \textit{mecánica} como ``reglas y procesos en un juego que lo vuelven funcional e interactivo.''(Traducción propia) \cite{lemarchandPlayfulProductionProcess2021}. Las mecánicas crean acciones llamadas \textit{game verbs}. Algunos ejemplos son saltar, caminar o disparar. Finalmente, define \textit{player activity} como ``las formas en las que un jugador usa un game verb en particular'' (Traducción propia) \cite{lemarchandPlayfulProductionProcess2021}. Por ejemplo, un jugador puede usar la barra espaciadora para ejecutar el \textit{game verb} de saltar. O puede ser un concepto mas abstracto, como tratar de encontrar un final diferente en una historia interactiva. \textit{player activity} es ``el resultado de combinar mecánicas, game verbs y narrativa con las acciones del jugador'' (Traducción propia) \cite{lemarchandPlayfulProductionProcess2021}. \todo{TODO: ultima frase es redundante. Reescribir}
\par Estos conceptos son importante porque para Lemarchand, un prototipo debe explorar un \textit{player activity} en particular para decidir si es interesante o no. Para ello, propone una serie de preguntas que un prototipo debe responder:
\begin{itemize}
    \item ¿Qué \textit{player activity} estoy prototipando?
    \item ¿Qué \textit{game verbs} estoy explorando?
    \item ¿Qué emocion evoca este \textit{player activity}?
    \item Este prototipo, ¿contesta alguna pregunta de game design? \todo{TODO: reescribir}
\end{itemize}
\bigbreak
\par Similar a Anderson, Lemarchand propone realizar tanto prototipos físicos como digitales. Los prototipos físicos son más rápidos de hacer, pero limitan el tipo de mecánicas que se pueden explorar. Por otro lado, los prototipos digitales son más lentos de hacer, pero permiten explorar una mayor variedad de mecánicas.
\par Playtesting es una parte importante del proceso. Lemarchand recomienda entregar los prototipos a otras personas lo antes posible, y observar cómo interactuan con ellos.
%
%
\subsubsection{Project goals}
\par El objetivo final de la etapa de \textit{Ideation} es definir los \textit{project goals}. La idea es que establecer objetivos claros para el proyecto ayuda a guiar el proceso de desarrollo en las siguientes etapas. Lemarchand separa los \textit{project goals} en dos categorias, \textit{experience goals} y \textit{design goals}. 
\bigbreak
\par Segun Lemarchand, un \textit{experience goal} es ``el tipo de experiencia que se le quiere dar al jugador, generalmente explicada en términos de la experiencia emocional.'' (Traducción propia) \cite{lemarchandPlayfulProductionProcess2021}. Ejemplos de esto puede ser la satisfacción de ganar o el miedo al jugar un juego de terror. La figura \ref{fig:x ejemplo de experience goals Lemarchand} muestra los \textit{experience goals} establecidos para Uncharted, uno de los juegos en los que trabajó el escritor.
%
\begin{figure}[H]
    \centering
    \includegraphics[scale=0.5]{experience_goals.png}
    \caption{``Experience goals'' del videojuego Uncharted. Extraída de \cite{lemarchandPlayfulProductionProcess2021}.}
    \label{fig:x ejemplo de experience goals Lemarchand}
\end{figure} 
%
\par Por otro lado, los \textit{design goals} son las mecánicas que complementan a los \textit{experience goals}. Algunas categorias comunes son:
\begin{itemize}
    \item La plataforma en la que se desarrollará el videojuego (PC, consola, móvil, etc.).
    \item Mecánicas, \textit{game verbs} y \textit{player activities}.
    \item El género del videojuego (plataforma, aventura, rol, etc.).
    \item La dirección artística del videojuego (estilo visual, música, etc.).
\end{itemize}
%
%
%  -- PRE PRODUCTION --
%
\subsection{Pre-production}
\par Lemarchand define a \textit{pre-production}(pre-producción en español) como la etapa donde se se establece el \textit{game design} y el plan a seguir durante la etapa de \textit{prorduction}. Al final de esta etapa, se obtienen 3 artefactos: un \textit{vertical slice}, un \textit{game design macro} y un \textit{schedule} \cite{lemarchandPlayfulProductionProcess2021}.
%
%
\subsubsection{Vertical Slice}
Un \textit{vertical slice} es una demo de alta calidad de un videojuego \cite{lemarchandPlayfulProductionProcess2021}. Tanto gráficos, sonido, mecánicas y narrativa deben estar lo suficientemente pulidos como para ser considerados parte del producto final. 
\par Esta demo se construye iterativamente a partir de intentos anteriores, siguiendo un \hyperref[sec:modelos_evolutivos]{modelo evolutivo}. Para ello, se definen una  o varias mecánicas, se crea un prototipo para probarlas, se realiza \textit{playtesting} y se evalúa el resultado. Al final de cada iteración se revisa si el prototipo cumple con los \textit{experience goals} definidos, y si es necesario, se realizan cambios. Este proceso se repite hasta que se obtiene un \textit{vertical slice} que cumple con los objetivos del proyecto.
\begin{figure}[h]
    \centering
    \includegraphics[scale=0.5]{vertical_slice_iteration.png}
    \caption{Proceso iterativo para la creación de un \textit{vertical slice}. Extraída de \cite{lemarchandPlayfulProductionProcess2021}.}
    \label{fig:x iteracion vertical slice}
\end{figure} 
\par Muchos juegos se construyen a partir de un patron repetitivo, llamado \textit{game loop}. Este patrón consiste en una serie de actividades que el jugador puede realizar, como atacar, disparar, saltar, etc. Luego, los \textit{game designers} crean distintas variaciones de estas actividades para mantener el interés del jugador. Este loop se debe ver representado en el \textit{vertical slice}, ya que es una parte fundamental del juego.
%
\paragraph{Playtesting}
\par Similar a otras metodologías de desarrollo, \textit{playtesting} es una parte fundamental del proceso. Este proceso debe realizarse tanto con otros miembros del equipo como con personas ajenas al proyecto. El objetivo es obtener feedback sobre el \textit{vertical slice} y evaluar si cumple con los \textit{experience goals} definidos. Para ello, Lemarchand propone una serie de prácticas:
\begin{itemize}
    \item \textbf{Minimizar la conversación con el jugador}: esto es útil para observar si el juego es intuitivo y entendible. 
    \item \textbf{Utilizar auriculares}: el sonido es una parte importante de la experiencia de juego, y puede guiar al jugador con el mismo.
    \item \textbf{Escribir los controles del juego en un papel}.
    \item \textbf{Escribir pistas para los problemas que ya se conocen}: es posible que la demo contenga algún bug  que requiera información adicional para que el jugador pueda continuar. En este caso, es recomendable escribir una pista para que el jugador pueda seguir jugando.
    \item \textbf{Sugerir que el jugador piense en voz alta}: el objetivo es obtener más información sobre la experiencia del jugador. Esto puede ayudar a identificar problemas que no son evidentes al observar al jugador.
    \item \textbf{Anotar las acciones del jugador}: escribir lo que el jugador va realizando en un papel puede ayudar a identificar tanto las partes interesantes del juego como las que generan problemas.
    \item \textbf{Realizar entrevistas luego de finalizar}: algunas preguntas recomendadas por Lemarchand son:
    \begin{itemize}
        \item ¿Cuál fue tu momento o interacción favorita?
        \item ¿Cuál fue tu momento o interacción que menos te gustó?
        \item ¿Hubo algo que quisieras hacer y que el juego no te permitiera? 
        \item Si pudieras cambiar algo del juego, ¿qué sería?
    \end{itemize}
\end{itemize}

\par Es importante analizar el feedback obtenido durante este proceso. Para ello, Lemarchand recomienda separar los datos en 3 categorías:
\begin{itemize}
    \item \textbf{Broken/must fix [roto/debe arreglarse]}: estos son problemas de \textit{game design} o bugs que impiden que el jugador tenga la experienca deseada, por lo que deben ser arreglados. Por lo general, la solución suele ser bastante obvia.
    \item \textbf{Question/maybe fix [pregunta/quizás deba arreglarse]}: estos son elementos que quizás no están funcionando correctamente, pero que requiere mayor investigación para determinar el por qué.
    \item \textbf{Suggestion/new idea [sugerencia/nueva idea]}: estas son sugerencias o ideas que pueden mejorar el juego, pero que no son necesarias para que el juego funcione correctamente. Por lo general, estas sugerencias se implementan en iteraciones posteriores.
\end{itemize}
%
%
\paragraph{Concentric development}
\par Lemarchand propone utilizar \textit{concentric development} durante la etapa de pre-producción. Este estilod de desarrollo consiste en comenzar por el núcleo del juego y luego ir expandiendo con sistemas que lo complementan. Lemarchand propone 3 capas:
\begin{itemize}
    \item \textbf{Primary mechanics}: estas son las mecánicas fundamentales del juego. Por ejemplo, en un juego de plataformas, las mecánicas primarias se refiere al movimiento del personaje y de la cámara. A diferencia de la etapa de \textit{Ideation}, estas mecánicas deben ser lo suficientemente pulidas como para ser parte del producto final. Esto significa crear el arte, sonido y programacion y game design necesarios para considerarse completas.
    \item \textbf{Secondary mechanics}: estas mecánicas consisten en los \textit{game verbs} mas importantes del juego, generalmente aquellos que completan el \textit{game loop}. Por ejemplo, en un juego de plataformas, las mecánicas secundarias pueden saltar sobre enemigos o recoger objetos.
    \item \textbf{Tertiary mechanics}: estas mecánicas son las que complementan a las primarias y secundarias. Siguiendo el ejemplo del juego de plataformas, las mecánicas terciarias pueden ser las que permiten al jugador interactuar con el entorno, como abrir puertas o activar interruptores.
\end{itemize}
\par La ventaja de trabajar de esta forma es que al final de pre-producción, se obtiene un \textit{vertical slice} que contiene las mecánicas fundamentales del juego, y estas mecánicas estan lo suficientemente pulidas como para decidir si el proyecto debe continuar o no. Este método de trabajo también es útil durante la etapa de producción. \todo{expandir ultima frase}
%
%
\subsubsection{Game Design Macro}
\par Lemarchand define un \textit{game design macro} como ``una matriz de ideas que representa un overview del diseño del juego.'' (Traducción propia) \cite{lemarchandPlayfulProductionProcess2021}. Esta matriz enumera los aspectos más importantes del juego, sin enfocarse en los detalles. Por ejemplo, puede incluir el número de niveles, o de personajes. Este artefacto es también una extensión de los \textit{project goals} definidos en la etapa de \textit{Ideation}. El \textit{game design macro} está compuesto de 2 partes: un documento llamado \textit{game design overview} y una tabla llamada \textit{macro chart}.
%
\paragraph{Game Design Overview} Este documento es una visión general del proyecto, y contiene una Introducción a los elementos más importantes del juego, incluyendo mecánicas, narrativa y dirección artística. \todo[size=\tiny]{tengo que poner referencia acá? Todo lo que esta en esa seccion es del mismo libro.}
%
\paragraph{Macro Chart} \todo{no me gusta el tamaño de la fuente} Este documento fue originalmente ideado por el desarrollador Mark Cerny, quien lo describe como ``qué tipo de gameplay va en cada lugar'' (Traducción propia) \cite{academyofinteractivearts&sciencesDICESummit20022012}. La idea es que esta tabla contiene los niveles del juego, las mecánicas que se utilizan en cada uno, y los fragmentos de historia que suceden al jugar esa sección.
%
\begin{figure}[h]
    \centering
    \includegraphics[width=\textwidth]{macro_chart.png}
    \caption{\textit{Macro chart} propuesto por Lemarchand. Extraída de \cite{lemarchandPlayfulProductionProcess2021}.}
    \label{fig:x macro chart}
\end{figure}
%
\par Las columnas \ref{fig:x macro chart} no son definitivas, sino que pueden variar dependiendo del proyecto. Sin embargo, Lemarchand pone énfasis en las siguientes:
\begin{itemize}
    \item \textbf{Player goal}: Lemarchand explica el concepto de \textit{autotelism}(``autolelismo'' en español). Este término, aplicado a videojuegos, se refiere a que cuando una persona juega un juego, establece objetivos para sí mismo basado en lo que el videojuego le permite hacer, lo que quieren hacer, lo que entienden que pueden hacer y lo que han experimentdado de juegos similares \cite{lemarchandPlayfulProductionProcess2021}. En base a esta información, en la columna de \textit{player goal} se debe escribir el objetivo que el jugador debe cumplir en cada sección del juego. Un ejemplo de esto puede ser ``llegar al final del nivel'' o ``derrotar al jefe''.
    \item \textbf{Design goal}: a diferencia del \textit{player goal}, este objetivo está orientado a los \textit{game designers}, y consiste en lo que se desea lograr en términos de mecánicas y narrativa. Por ejemplo, ``introducir la mecánica de saltar'' o ``presentar un nuevo personaje''.
    \item \textbf{Emotional beat}: Esta columna describe lo que se desea que el jugador sienta al jugar esa sección del juego. Por ejemplo, ``sorpresa'' o ``alegría''.
\end 
{itemize}
\par Lemarchand llama \textit{sequences}(``secuencias'' en español) a las filas de la tabla, donde cada una representa una sección del juego. Lemarchand propone una serie de conceptos que ayudan a definir su orden: \todo{reescribir} %TODO: reescribir
\begin{itemize}
    \item \textbf{Secuenciar gameplay}: para Lemarchand, un buen punto de partida es considerar el orden en que se introducen las mecánicas del juego \cite{lemarchandPlayfulProductionProcess2021}. Por lo general, los \textit{game designers} crean secuencias que añaden mecánicas y dificultad progresivamente.
    \item \textbf{Secuenciar narrativa}: otra herramienta para definir el orden de las secciones es considerar la narrativa del juego. Por ejemplo, es necesario presentar a los personajes, el mundo en el que se desarrolla la historia, y los conflictos que deben resolver. Esto puede ayudar a definir el orden de las secciones del juego.
    \item \textbf{Secuenciar espacio}: otra forma de ordenar las secciones es considerar el espacio en el que suceden, especialmente si el juego está separado en niveles. Lemarchand destaca que crear espacios es una tarea que requiere mucho trabajo, por lo que es importante definir su orden y buscar maneras de reutilizar espacios.
\end{itemize}
\par Por otro lado, es importante considerar la duración de cada sección. Por ejemplo, para juegos grandes como \textit{Uncharted}, cada sección representa 15 minutos de juego. Para proyectos más pequeños cada sección puede durar entre 1 y 5 minutos. \todo{no se donde meter esto} %TODO: no se donde meter esto
\par Finalmente, para completar el \textit{macro chart} se deben añadir el resto de pantallas importantes, como el menú principal, el menú de pausa y opciones, y las pantallas de \textit{game over} \footnote{Pantalla que aparece cuando el jugador pierde en el juego.}, de carga y créditos.
%
%
\subsubsection{Schedule}
\par Durante pre-producción, el proceso de desarrollo es iterativo y no sigue un cronograma fijo. En cambio, para la etapa de producción, es imperativo programar el progreso de desarrollo mediante un \textit{schedule} (cronograma en español).
\par Para comenzar este proceso, Lemarchand recomienda calcular lo que llama \textit{person-hour}. Este recurso consiste en la cantidad de horas que un miembro del equipo dedica al proyecto. El valor se calcula en base a la cantidad de semanas que se dedicarán al proceso de producción, el número de miembros del equipo y la cantidad de horas que cada uno dedica durante la semana.
\begin{equation}
    T = N \times P \times W
\end{equation}

donde:
\begin{itemize}
    \item $T$ = Total de \textit{person-hours} disponibles para el proyecto.
    \item $N$ = Número de miembros del equipo.
    \item $P$ = Horas-persona que cada miembro trabajará en una semana promedio.
    \item $W$ = Número de semanas dedicadas a producción.
\end{itemize}
\par Por ejmplo, para un equipo de 3 personas, que dedican 20 horas a la semana, durante 10 semanas, el total de \textit{person-hours} disponibles es:
\begin{equation}
    T = 3 \times 20 \times 10 = 600
\end{equation}
\par Luego, este número se compara con el número de \textit{person-hours} que se estima que se necesitan para completar todos los items del \textit{macro chart}, además de otras tareas como escribir dialogos, participar en reuniones, o realizar \textit{playtests}.
\par Lemarchand recomienda comenzar con una tabla que contenga las tareas a realizar en conjunto con las siguientes columnas:
\begin{itemize}
    \item \textbf{Priority (prioridad)}: Esta columna indica la prioridad de cada tarea. Lemarchand propone 3 niveles:
    \begin{itemize}
        \item Priority 1 (prioridad 1): tareas obligatorias, que deben completarse para que el juego funcione correctamente. Por ejemplo, el movimiento del personaje, junto a sus animaciones, controles y sonidos.
        \item Priority 2 (prioridad 2): tareas que son importantes, pero que no son necesarias para que el juego funcione correctamente. Por ejemplo, los \textit{game verbs} utilizados por el jugador para interactuar con el mundo, como ``hablar'' o ``tirar objeto''.
        \item Priority 3 (prioridad 3): tareas que son opcionales, pero que pueden mejorar la experiencia del jugador. Por ejemplo, animaciones adicionales u objetos del entorno de juego.  
    \end{itemize}
    \item \textbf{Hours estimated (horas estimadas)}: Esta columna indica la cantidad de horas que se estima que se necesitan para completar cada tarea. Para Lemarchand, esta columna sólo puede tomar los valores de 1, 2, 4 u 8 horas. Aquellas tareas pequeñas se pueden agrupar en una sola, y las tareas que tomen más de 8 horas deben dividirse en subtareas más pequeñas. El objetivo es reducir la incertidumbre de las estimaciones, y que cada tarea sea lo suficientemente pequeña como para que se pueda estimar con precisión. 
    \item \textbf{Member assigned (miembro asignado)}: Esta columna indica el miembro del equipo que se encargará de cada tarea. Lemarchand recomienda asignar tareas a un solo miembro, ya que esto ayuda a evitar confusiones y conflictos.
\end{itemize} 
%
\begin{figure}[h]
    \centering
    \includegraphics[width=\textwidth]{schedule_example.png}
    \caption{Ejemplo de un \textit{schedule}. Extraída de \cite{lemarchandPlayfulProductionProcess2021}.}
    \label{fig:x schedule ejemplo}
\end{figure}
%
\par Cabe destacar que esta comparación entre \textit{person-hours} disponibles y las horas estimadas para completar las tareas no es una estimación precisa, sino que es una forma de tener una idea general de si el proyecto es viable o no.

