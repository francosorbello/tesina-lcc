\par Mas allá de la metodología elegida por los desarrolladores, los videojuegos suelen pasar por las siguientes fases:
\begin{enumerate}
    \item \textbf{Iniciación}: La iniciación es la fase inicial del desarrollo. En esta, se crea el concepto del videojuego a desarrollar, mediante herramientas como las lluvias de ideas o feedback proveniente en proyectos anteriores \cite{shresthaGameDevelopmentLifecycle2023,widjajaUtilizingGameDevelopment2024,ramadanGameDevelopmentLife2013}.
    \item \textbf{Pre-producción}:  En esta etapa se prueba la idea presentada en la iniciación. Se define el género del juego, las mecánicas e historia que lo conforman, además del aspecto visual y sonoro a seguir. Esta información se plasma en un documento de diseño (GDD). Además, se desarrollan prototipos jugables para probar si la idea es divertida y vale la pena desarrollarla por completo \cite{shresthaGameDevelopmentLifecycle2023,widjajaUtilizingGameDevelopment2024,ramadanGameDevelopmentLife2013}.
    \item \textbf{Producción}: Esta es la etapa más larga y costosa del desarrollo. Consiste en la creación de código y assets que constituyen el juego \cite{shresthaGameDevelopmentLifecycle2023,widjajaUtilizingGameDevelopment2024,ramadanGameDevelopmentLife2013}.
    \item \textbf{Testing}: En esta etapa se realiza una prueba en profundidad de las funcionalidades del juego, con el objetivo de encontrar y arreglar errores.
    \item \textbf{Lanzamiento}:  Una vez el juego está terminado, se lanza al mercado.
    \item \textbf{Post-lanzamiento}: El desarrollo no termina con el lanzamiento, sino que es común continuar actualizando el juego, con parches y nuevo contenido.
\end{enumerate}