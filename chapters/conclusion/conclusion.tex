\par Para finalizar este trabajo se presentan las conclusiones derivadas del desarrollo práctico realizado. En particular, se analizan las metodologías utilizadas y el impacto que tuvieron en el proyecto.
\bigbreak
\par Como metodología de desarrollo empleada, se decidió utilizar un \textbf{modelo en cascada} basado en los textos de Anderson\cite{andersonProductionPointHow2023} y Lemarchand\cite{lemarchandPlayfulProductionProcess2021}. Por un lado Anderson propone el siguiente sistema de trabajo:
\begin{enumerate}
    \item Pre-producción: esta etapa comienza con el desarrollo de varios prototipos. Para ello, Anderson recomienda participar en game jams o realizar juegos en periodos cortos de tiempo. Una vez que se encuentra una idea prometedora, se realiza un \textit{vertical slice} para validar si el juego tiene potencial para convertirse en un producto completo.
    \item Production Point: en esta etapa se valida el estado del proyecto utilizando el \textit{vertical slice}. Se decide si volver a pre-producción, pasar a producción o cancelar el juego. 
    \item Producción: esta etapa consiste en crear contenido para el juego. Debido a que el progreso tiene un formato lineal, Anderson recomienda utilizar herramientas como Kanban para medir el progreso.
    \item Alpha y Beta: en esta etapa se realizan tareas corregir bugs, localizar el juego o implementar otras opciones necesarias. El objetivo es pulir el proyecto para lanzarlo al mercado.
    \item Lanzamiento: como indica el nombre, esta etapa consiste en publicar el juego. Requiere algunas tareas como crear trailers y otros \textit{recursos} de marketing.
    \item Post-Lanzamiento: en esta etapa, se corrigen errores que se encuentren al lanzar el juego y se planea cómo expandir el juego a futuro, en caso de que se considere viable.
\end{enumerate}

\noindent
Por otro lado, Lemarchand propone el proceso a continuación:
\begin{enumerate}
    \item Ideación: esta etapa comienza pensando ideas utilizando herramientas de \textit{brainstorming}, que luego se validan creando prototipos. La etapa finaliza definiendo los \textit{project goals}, una serie de objetivos que guían el proceso de desarrollo en las siguientes etapas.
    \item Pre-producción: esta fase consiste en iterar las ideas obtenidas durante ideación mediante el desarrollo de un \textit{vertical slice}. Los \textit{project goals} se utilizan para asegurarse que el proyecto vaya por buen camino. Además, introduce el uso de \textit{playtesting} para validar el progreso. Al finalizar esta etapa, se crea un \textit{game design macro} y un \textit{schedule}. El primero es una matriz que enumera los aspectos más importantes del proyecto como los niveles y las mecánicas que se utilizan en ellos, mientras que el segundo contiene una lista de tareas necesarias para terminar el juego.
    \item Producción: La siguiente etapa consiste en crear el contenido del juego basado en lo establecido en el \textit{game design macro} y el \textit{schedule}. También se realiza una fase de Alpha y Beta, con el objetivo de encontrar errores y asegurarse la calidad del contenido creado.
    \item Post-producción: el desarrollo finaliza con esta fase, que consiste mayormente en arreglar errores y pulir y balancear el juego, con el objetivo de tener un producto que pueda lanzarse al mercado. 
\end{enumerate}
\par Debido a que ambos autores presentan varias similitudes, se decidió crear un modelo que unifique sus ideas, replicando el sistema en cascada y tomando las herramientas que se consideren beneficiosas para el proyecto. Así, el proceso final de desarrollo consistió en:
\begin{enumerate}
    \item Iniciación: esta etapa se basa en la pre-producción de Anderson y la ideación de Lemarchand. Consiste en probar varias ideas, realizando prototipos y participando en game jams. Al finalizar se establecen los \textit{project goals} del juego.
    \item Pre-producción: se unifican las etapas de Anderson y Lemarchand del mismo nombre. Durante esta fase se iteran las ideas obtenidas durante la iniciación hasta obtener un \textit{vertical slice}. Se pone fuerte enfoque en realizar \textit{playtesting}. Además, se crea un \textit{game design macro} para organizar el trabajo de la siguiente etapa.
    \item Producción: siguiendo lo establecido por Lemarchand y Anderson, se crea el contenido del juego utilizando el \textit{game design macro}.
    \item Post-producción: esta etapa combina el Alpha y Beta de Anderson con la post-producción de Lemanchard. Se destina tiempo a pulir el proyecto, eliminar bugs y completar el juego.
    \item Lanzamiento: esta fase es similar a la propuesta por Anderson. Se crean los \textit{assets} requeridos para marketing y se publica el juego.
\end{enumerate}

%añadir que utilicé kanban durante distintas etapas para organizar el trabajo en particular, pero que el big picture fue en cascada

%explicar las ventajas de dividir el trabajo en cada etapa. explicar que es inevitable tener una preproduccion desestructurada, que es parte del proceso.

\par Este método de desarrollo trabajo tuvo sus pros y sus contras. Por un lado, se destaca la separación estricta de las etapas, especialmente entre pre-producción y producción. Durante la preproducción se establecen las bases fundamentales del juego, incluyendo el diseño del sistema y el prototipado de las mecánicas principales. En cambio, la etapa de producción se centra en la creación de contenido apoyándose en estas bases previamente definidas. Esta separación implica que no se introducen nuevas mecánicas durante la producción, lo cual contribuye a mantener el proyecto controlado y viable, reduciendo riesgos asociados a cambios tardíos en el diseño. Sin embargo, una desventaja relevante de este enfoque es que la etapa de pre-producción puede extenderse considerablemente, ya que no es posible avanzar a producción hasta que todas las mecánicas estén completamente definidas y validadas, lo que puede impactar negativamente en los tiempos totales de desarrollo.
\par Por otro lado, se encontraron dificultades para planificar el trabajo realizado durante pre-producción. Esto se debe a que se trata de una fase de prototipado constante en la que la experimentación y la iteración son fundamentales. Esta naturaleza exploratoria dificulta la definición clara de tareas y la estimación de tiempos, lo que puede afectar la organización del desarrollo, aun cuando dicho proceso sea inherente a los objetivos de esta etapa. 
\bigbreak
\par Un punto importante a destacar es la diferencia entre el trabajo individual y en equipo. Una ventaja de trabajar de forma individual es que se evitan los problemas que aparecen al trabajar con otras personas, como falta de comunicación o dependencia entre roles. En un contexto colaborativo, sería necesario establecer sistemas adicionales de seguimiento y coordinación para asegurar que cada etapa del desarrollo avance de forma coherente. En este sentido, una ventaja de la metodología en cascada es que cada fase puede organizarse de manera independiente y adaptarse a las necesidades específicas del desarrollador, permitiendo incluso la integración de otras metodologías (como Agile o Kanban) dentro de cada etapa.

\bigbreak
\par En conclusión, se considera que la metodología utilizada resultó de gran ayuda para estructurar el desarrollo del proyecto, especialmente gracias a la clara separación entre etapas y al control del alcance durante la producción. No obstante, la experiencia obtenida también indica que la etapa de preproducción representa un desafío significativo, particularmente en términos de organización, planificación y duración. Queda como una línea de trabajo futura la investigación de estrategias y metodologías que permitan optimizar el proceso de preproducción, manteniendo su carácter iterativo y experimental, pero mejorando su eficiencia y previsibilidad.