\par Este trabajo final trajo consigo varias lecciones sobre el desarrollo de videojuegos y del software en general. Uno de los aspectos críticos que se evidenció durante el desarrollo fue la importancia de establecer un alcance de proyecto razonable desde las etapas iniciales. Este trabajo presenta deficiencias en este aspecto, pues se subestimó la complejidad y el tiempo requerido para completar ciertas funcionalidades, lo que resultó en retrasos y la necesidad de realizar ajustes constantes al plan original.
\bigbreak
\par En cuanto a la metodología de desarrollo empleada, la utilización de un modelo en cascada adaptado resultó beneficiosa para este contexto particular. La estructura secuencial de ideación, pre-producción, producción y lanzamiento proporcionó un marco claro de trabajo. Se recomienda mantener esta estructura en futuros proyectos de desarrollo individual, ya que proporciona hitos claros y permite evaluar el progreso de manera objetiva.
\bigbreak
\par Finalmente, se considera que hay varios temas que presentan nuevas avenidas de investigación. Si bien este trabajo se ha centrado en la experiencia del desarrollo individual, los aprendizajes obtenidos plantean interrogantes interesantes sobre su escalabilidad en un contexto de trabajo en equipo. Otro tema a investigar sería la implementación de metodologías ágiles durante producción. Para este proyecto solo se utilizó Kanban, pero esto fue debido a las limitaciones de tiempo. Queda para futuro investigar el uso de herramientas más avanzadas durante esta etapa.