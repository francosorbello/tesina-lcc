La industria de los videojuegos es destacada por su constante crecimiento y rentabilidad. No solo esto, si no que requiere una mano de obra altamente especializada. Programadores, músicos, artistas y escritores forman equipos para crear obras de arte interactivas. Sin embargo, la gestión de equipos tan diversos y la coordinación de múltiples tareas presenta un desafío considerable a la hora de llevar a cabo un proyecto de desarrollo de videojuegos.
\bigbreak
Este desafío aumenta al desarrollar un juego desde la perspectiva de un solo dev, es decir que el trabajo es llevado a cabo por una sola persona. El trabajo que comúnmente caería en otros miembros del equipo, como artistas o músicos, pasa a ser responsabilidad del desarrollador. Es por esto que este trabajo busca una metodología que permita llevar a cabo ese proceso de forma ágil y accesible.
\bigbreak
En general, en videojuegos no parece haber una línea clara a la hora de definir una metodología de trabajo. Por un lado, las metodologías ágiles, explicadas en el capítulo 2, parecen ser populares en el desarrollo de videojuegos. En 2013, Koutonen and Leppänen entrevistaron a 20 estudios finlandeses, donde todos menos uno indicaron que aplicaban Agile en al menos una de las fases de desarrollo \cite{koutonenHowAreAgile2013}.
En 2016, Politowski et al. entrevistaron a 58 desarrolladores brasileños y también encontraron un alto uso de metodologías ágiles \cite{politowskiSoftwareEngineeringProcesses2016}.
De forma similar, en 2021 McKenzie et al. entrevistaron a 8 estudios de Nueva Zelanda, los cuales indican utilizar alguna estrategia ágil, como Scrum o Kanban \cite{mckenzieAgileNotAgile2021}.
Finalmente, en 2025 Saarentaurus entrevistó a 5 productores de empresas de videojuegos finlandesas, y todos indicaron que utilizaban Agile en sus proyectos \cite{saarentausAgileScrumMethods2025}.
Estos productores trabajaban en empresas distintas, y todos tenían más de 5 años de experiencia en la industria. Cabe destacar que el estudio se realizó en empresas con más de 100 empleados \cite{saarentausAgileScrumMethods2025}.
\break
Si bien el tamaño de las muestras en cada estudio es pequeño, se puede notar que a lo largo de los años las metodologías ágiles han mantenido un nivel de relevancia en el desarrollo de videojuegos
\bigbreak
Otro fenómeno común al desarrollo es que los estudios terminen utilizando lo que se conoce como metodologías ad-hoc, que consiste en editar otros métodos de desarrollo con el objetivo de ajustarlas a las necesidades de la empresa. McKenzie et al. destacó que si bien los desarrolladores entrevistados indican usar metodologías ágiles, la mayoría realizan modificaciones a Scrum, como eliminar ciertas prácticas, o cambiar la duración de otras \cite{mckenzieAgileNotAgile2021}.
Estos cambios se deben a que es necesario adaptar Scrum a las particularidades del desarrollo de videojuegos \cite{mckenzieAgileNotAgile2021,bartoszScrumVideoGames2023}.
Como se explicó anteriormente, en la creación de un juego participan personas de áreas totalmente distintas, desde programación hasta música. Además, como los juegos son un producto de entretenimiento, uno de los requerimientos del software es que sea “divertido”. Esto es un problema a la hora de planificar ya que es difícil establecer objetivos basados en algo tan subjetivo como la diversión \cite{mckenzieAgileNotAgile2021,bartoszScrumVideoGames2023}. Es por esto que las empresas terminan realizando cambios a la metodología que adoptan.
\bigbreak
A esto se suma que algunos autores han propuesto metodologías específicas para el desarrollo de videojuegos. En 2024, Widjaja et al. desarrollaron un juego utilizando la metodología propuesta por Ramadan y Widyani \cite{widjajaUtilizingGameDevelopment2024}. Este sistema propone un modelo similar a modelos evolutivos pero incluye criterios de calidad relacionados con crear un producto “divertido” \cite{ramadanGameDevelopmentLife2013}. De forma similar, Harman propone un framework que separa distintas etapas del desarrollo en un método de cascada, y luego utiliza conceptos de metodologías ágiles en algunas de ellas \cite{harmanDevelopmentMethodologiesGame2023}.
\bigbreak
Hay un tercer grupo de desarrolladores que no es mencionado en los estudios anteriormente nombrados; los solo devs, o equipos pequeños (2-5 personas). La dinámica de desarrollo cambia debido a la poca cantidad de personas involucradas. No hay mucha investigación al respecto, pero de las fuentes obtenidas se puede observar el uso de metodologías ad-hoc, lo cual es congruente con lo mencionado anteriormente. Similar a Radaman, Widyani y Harman, Anderson separa el desarrollo en etapas de un método en cascada. Además, utiliza Kanban para organizar las tareas \cite{andersonProductionPointHow2023}. Robinson-Yu presenta una idea parecida, con el agregado de utilizar una versión simplificada de Scrum para llevar a cabo sus tareas \cite{gamedevelopersconferenceCraftingTinyOpen2020}.
\bigbreak
En este documento se estudian estas 3 vertientes, con el objetivo de formar una metodología que permita desarrollar un juego como solo dev, tomando los elementos que se consideren útiles y eliminando los que dificulten el trabajo.



