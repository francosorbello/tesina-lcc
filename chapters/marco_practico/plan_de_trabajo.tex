\par Como parte de la tesina, se propone un plan de trabajo para el desarrollo de un videojuego que combine las metodologías estudiadas. Este plan se forma considerando los siguientes aspectos:
\begin{itemize}
    \item El plan debe utilizar lo estudiado en el marco teórico.
    \item El plan debe formarse considerando que el proyecto lo lleva a cabo un solo desarrollador.
\end{itemize}
\par Para ello, se divide el proceso en 5 etapas: iniciación, pre-producción, producción, post-producción y lanzamiento. Estas etapas se separan utilizando el método en cascada, es decir que no se puede continuar de una fase a otra hasta completar la anterior. A continuación se detallan las etapas y sus acciones. 
\paragraph{Iniciación} El proceso comienza con \textbf{1 mes} de prototipado, donde se desarrollará \textbf{1 prototipo por semana}. El objetivo es probar distintas ideas hasta encontrar un proyecto con potencial para convertirse en un juego completo.
\paragraph{Pre-producción} Esta etapa toma uno de los prototipos creados durante la iniciación y lo expande para crear un juego completo. Se definen los \textit{project goals} y se crean varios prototipos hasta llegar a un \textit{vertical slice}. También se destina tiempo a realizar sesiones de \textit{playtesting} para evaluar los prototipos. Estas sesiones combinan las grabaciones de Anderson con entrevistas a los jugadores, siguiendo el enfoque de Lemarchand. Se busca obtener feedback sobre la jugabilidad, la mecánica y la experiencia del usuario. Al final de esta etapa se crea un \textit{game design macro} del juego, que define las secciones del juego y su duración. Este \textit{macro} se divide en secciones de un largo proporcional al total del juego, siguiendo el enfoque de Lemarchand y Anderson.
\paragraph{Producción} Durante esta etapa se expanden mecánicas definidas en pre-producción, añadiendo contenido y refinando la jugabilidad. Debido a que esta etapa es más estructurada, se utilizan conceptos de Scrum y Kanban para organizar el trabajo. Se establecen sprints de \textbf{1 semana} con un \textit{burndown chart} para monitorear el progreso. Las tareas se organizan en un tablero Kanban, donde se pueden mover las tareas entre las columnas de \textit{To Do}, \textit{In Progress} y \textit{Done}.
\paragraph{Post-producción} Para finalizar el desarrollo, se destina tiempo a realizar un sistema de pruebas \textit{alpha} y \textit{beta}, siguiendo el enfoque de Anderson y Lemarchand. Durante la fase de \textit{alpha}, se filtra el contenido del juego, eliminando lo que no es necesario y refinando las mecánicas. En la fase de \textit{beta}, se trabaja en arreglar bugs y mejorar la experiencia de usuario. Esta fase requiere un grupo grande de testers que prueben el juego, por lo que parte del trabajo es encontrar personas interesadas en realizar \textit{playtesting}.
\paragraph{Lanzamiento} Finalmente, se prepara el proyecto para su lanzamiento. Durante esta etapa, se realizan las últimas correcciones, se crea un \textit{presskit} y se publica el juego en las plataformas deseadas.

\begin{figure}[H]
\centering
\includegraphics[width=\textwidth]{practica_cascada.png}
\caption{Etapas del plan de trabajo, utilizando método en cascada. Creada por autor}
\label{fig:x practica_cascada}
\end{figure}