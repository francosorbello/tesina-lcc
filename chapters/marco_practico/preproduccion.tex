\par Esta etapa fue por lejos la más larga e importante del proyecto. Durante esta etapa se construyeron las bases que luego representaron el contenido del juego. El primer paso fue analizar los proyectos realizados durante la etapa de Iniciación. Del feedback recibido se extrajeron las siguientes conclusiones:
\begin{itemize}
  \item Los jugadores presentan interés en el aspecto narrativo de los juegos.
  \item Los jugadores disfrutaron del diseño de niveles, en particular del prototipo 4.
  \item Los jugadores también muestran interés en el diseño sonoro y la inmersión que genera.
  \item Los jugadores detectan discrepancias en el aspecto visual de los juegos. 
\end{itemize}
\par Basado en esta información, se decidió continuar con el \hyperref[sec:prototype_3]{prototipo 3}, expandiendo la idea original de explorar un laberinto submarino con un nuevo sistema de movimiento, más niveles y nuevos personajes. Se eligió este proyecto porque destaca en los aspectos positivos recibidos en el feedback. Este prototipo recibió críticas positivas por su historia, y el concepto del laberinto submarino permite crear niveles similares a los desarrollados en el prototipo 4.
\subsubsection{Project goals}
\par Siguiendo lo indicado por Lemarchand, se establecieron una serie de objetivos, llamados \textit{project goals}. Estos se separan en 2 categorías:
\paragraph{Experience goals} Se refiere al tipo de experiencia que se le quiere dar al jugador. Para este juego, los \textit{experience goals son}:
\begin{itemize}
  \item La narrativa y la exploración del entorno son el foco del juego.
  \item La narrativa presenta historias humanas en un entorno inhumano: el objetivo es que el jugador explore espacios misteriosos e ilógicos, pero que las historias que sucedan en ese universo se sientan cercanas.
  \item La exploración debe ser divertida y satisfactoria: el jugador debe tener herramientas que lo motiven a descubrir el universo.
\end{itemize}
\paragraph{Design goals} estas son las mecánicas que complementan a los \textit{experience goals}:
\begin{itemize}
  \item El juego se desarrolla para la plataforma Windows.
  \item El juego es del género "plataformero".
  \item La dirección artística utiliza elementos del género de terror para su ambientación. El diseño sonoro está inspirado en juegos de terror de las primeras consolas en 3D, como la Playstation 1.
  \item El juego está dirigido a adultos jóvenes que juegan novelas visuales o consumen juegos de terror. 
\end{itemize}
\bigbreak
\par El proyecto cambió mucho durante el proceso, producto de la constante iteración de ideas y mecánicas. Inicialmente, se probó siguiendo el esquema de "pantalla partida" presente en el prototipo 3, donde la mitad de la pantalla está destinada a mostrar el juego y la otra mitad narra los eventos que suceden en texto.

\begin{figure}[H]
\centering
\includegraphics[width=\textwidth]{prototype_3_to_preprod.png}
\caption{Arriba: comparación entre el prototipo 3 (arriba) y el primer prototipo de pre-producción (abajo). Creada por autor.}
\label{fig:x prototype3_to_preprod}
\end{figure}

\begin{figure}[H]
\centering
\includegraphics[width=\textwidth]{visuals_v2.png}
\caption{Captura del segundo prototipo de pre-producción, con un estilo visual similar al que se encuentra en el juego final. Creada por autor.}
\label{fig:x visual_prototype_2}
\end{figure}

\par En el prototipo original, el jugador se mueve seleccionando una dirección de una lista de opciones. Este sistema se reemplazó con la habilidad de navegar el espacio utilizando el teclado y mouse. Sin embargo, este esquema resultó incómodo de manejar en pantalla partida, por lo que eventualmente se mantuvo solamente la pantalla de juego \ref{fig:x visual_prototype_3}.

\begin{figure}[H]
\centering
\includegraphics[width=\textwidth]{visuals_v4.png}
\caption{Captura del prototipo de pre-producción, que elimina la pantalla compartida y mantiene la pantalla de juego. Creada por autor.}
\label{fig:x visual_prototype_3}
\end{figure}

\par Siguiendo el consejo de Anderson, se crearon \textit{mockups} para definir el estilo visual del juego. Se buscaron estéticas llamativas, pero que fueran accesibles para una persona sin entrenamiento artístico. Eventualmente, se decidió utilizar una estética 1-bit y aprovechar efectos relacionados con la programación, como shaders y partículas.

\begin{figure}[H]
\centering
\includegraphics[width=\textwidth]{mockup.png}
\caption{\textit{Mockups} de arte desarrollados para el proyecto. Creada por autor.}
\label{fig:x mockup}
\end{figure}

\par El diseño del juego también sufrió grandes cambios durante esta etapa. Originalmente, se deseaba crear niveles interconectados entre sí, con secretos y caminos alternativos. Sin embargo, debido a las limitaciones de tiempo, el producto final consiste en niveles lineales, con un solo camino posible entre objetivos. 

\par Para apoyar este sistema de niveles, se desarrolló un sistema de inventario y habilidades que permitían al jugador utilizar varios ítems y poderes para acceder a nuevas partes del mapa. De nuevo, estos sistemas fueron rediseñados o eliminados para cumplir con las restricciones de tiempo del proyecto. Además, el juego originalmente contenía una mecánica de oxígeno, similar al prototipo en el que estaba basado. Esta mecánica pasó por varias iteraciones, pero eventualmente también se eliminó.
%
%
\subsection{Playtesting y Vertical Slice}
\par Lo importante a destacar es que esta etapa está marcada por un proceso fuerte de prueba y error, donde ideas pasan por varias etapas hasta llegar a un producto divertido o interesante. Esto dificulta medir el progreso del proyecto durante esta etapa. Sin embargo, tanto Lemarchand como Anderson proponen realizar playtesting para contrarrestar esta incertidumbre.
\par Durante la pre-producción se realizaron 2 rondas de playtesting, cada una asociada a un vertical slice diferente. El proceso de playtesting se basó en lo establecido por Lemarchand y Anderson.
\begin{enumerate}
  \item Los jugadores obtenían acceso a un ejecutable del juego, el cual descargaban desde la página web itch.io. La ventaja de esto es que es posible restringir el acceso mediante contraseña, de forma que solo los participantes tuvieran acceso al juego.
  \item Si bien no era obligatorio, los jugadores podían subir grabaciones de la pantalla del juego. Otra opción era establecer una reunión virtual con el jugador, para que compartiera pantalla mientras jugaba.
  \item Al finalizar la demo, el jugador era redireccionado a una encuesta. Este formulario seguía los lineamientos establecidos por Lemarchand. Primero se presentaban preguntas en formato escala de Likert, evaluando distintos sistemas del juego. Luego, se encontraba una sección de preguntas abiertas, donde el jugador podía opinar sobre distintos temas.
\end{enumerate}

\begin{figure}[H]
\centering
\includegraphics[width=\textwidth]{playtest_form_example.png}
\caption{Algunas de las preguntas realizadas durante el proceso de playtesting. Creada por autor.}
\label{fig:x }
\end{figure}

\subsubsection{Vertical Slice 1 (Mecánicas)}
\par El primer vertical slice fue creado para probar las mecánicas del juego. En particular, se quería analizar si el juego era intuitivo y divertido de jugar. El prototipo consistía una serie de niveles con distintos obstáculos a superar. La encuesta también preguntaba sobre el apartado artístico y sonoro del juego. Este prototipo también se mostró en un evento presencial, donde también se obtuvo feedback del juego. En total, 10 personas jugaron el juego y 2 completaron la encuesta, además de grabar su pantalla mientras jugaban.
\par En general, la respuesta de los jugadores fue positiva, pero se marcaron algunas correcciones a realizar. Varias veces se mencionó que algunos elementos visuales no correspondían con el estilo artístico. Además, algunos puzzles no eran divertidos de resolver. Estos problemas se arreglaron en versiones subsecuentes del juego.
\bigbreak
\par Se puede acceder a la encuesta,las grabaciones y a las respuestas desde el siguiente link:
\url{https://drive.google.com/drive/folders/1PkKxrMDSrUW9F5Ledvs-8v5aCfR2Op5h?usp=sharing}

\subsubsection{Vertical Slice 2 (Narrativa)}
\par Se creó un segundo vertical slice para observar cómo los jugadores respondían a la historia del juego. Esta ronda de playtest fue únicamente a través de internet, y en total participaron 5 personas. Tres personas enviaron grabaciones de su pantalla, mientras que la cuarta persona compartió su pantalla mientras jugaba, utilizando Discord.
\par En general, la historia del juego recibió feedback positivo. En cambio, se obtuvieron nuevos comentarios respecto a algunos elementos del diseño, indicando en particular que era incómodo controlar el juego con teclado. Similar al prototipo anterior, estos errores se arreglaron para la versión final.
\bigbreak
\par Se puede acceder a la encuesta,las grabaciones y a las respuestas desde el siguiente link:
\url{https://drive.google.com/drive/folders/1SMbRcKXLjdR1Aya_PXd08zRW3UqOldf0?usp=sharing}
\subsection{Game Design Macro}
\par Para finalizar la etapa de pre-producción, se armó un \textit{game design macro}, siguiendo los lineamientos establecidos por Lemarchand. Este documento detalla cada nivel, con las mecánicas y los personajes que le corresponden. Este documento luego facilitó la creación de tareas durante la fase de producción.
\begin{figure}[H]
\centering
\includegraphics[width=\textwidth]{game_design_macro.png}
\caption{Captura de pantalla del \textit{Game Design Macro}. Creada por autor.}
\label{fig:x game_design_macro}
\end{figure}
\par El documento completo se puede encontra en el siguiente link: \url{https://docs.google.com/spreadsheets/d/10gvpafwtd-RiVymKt61cvXWMFo6WI0LQwEtWwTZCQOY/edit?usp=sharing}

