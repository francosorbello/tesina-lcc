\par La etapa de producción consistió mayormente en completar el juego llevando a cabo las siguientes tareas:
\begin{enumerate}
    \item Diseñar los niveles establecidos en el \textit{game design macro}.
    \item Añadir diálogos y cinemáticas establecidos en el \textit{game design macro}.
    \item Añadir otras funcionalidades importantes, como menú de pausa y menú de configuraciones.
    \item Crear los \textit{assets} para promocionar el juego, particularmente los logos e imágenes requeridas por \textit{itch.io}\cite{DownloadLatestIndie}, la página donde se publicó el proyecto.
\end{enumerate}
\par Debido a que esta etapa tiene un formato de trabajo más estructurado que las anteriores, se utilizó Kanban para medir el progreso de las tareas.
\begin{figure}[H]
\centering
\includegraphics[width=\textwidth]{trello_production_phase_v2.png}
\caption{Kanban utilizado durante la fase de producción. Creada por autor.}
\label{fig:x trello_production_phase}
\end{figure}
\par Este tablero sigue el formato clásico establecido por Kanban, con una columna de ``Por Hacer (TODO)'', ``En Proceso (DOING)'', y ``Finalizado (DONE)''. Se añadió una columna extra llamada ``Nice To Have'', que incluía tareas opciones que solo se llevarían a cabo en caso de tener tiempo extra.
\par Luego, se creó una tarea para cada nivel y cinemática. A medida que se avanzó en el proyecto, se añadieron otras tareas, como arreglo de errores o correcciones a algunos de los diálogos.
\bigbreak
\par La última tarea de esta fase fue crear los \textit{assets} requeridos para publicar el juego, incluyendo un banner, un logo, y el texto que promociona las \textit{features} del juego. El juego se publicó en la tienda online \textit{itch.io}\cite{DownloadLatestIndie}, un servicio creado tanto para publicar como vender y comprar juegos.
\begin{figure}[H]
\centering
\includegraphics[width=\textwidth]{marketing_game.png}
\caption{Captura de pantalla del juego en la tienda \textit{itch.io}\cite{DownloadLatestIndie}. Creada por autor.}
\label{fig:x marketing_game}
\end{figure}