En total, el desarrollo tomó aproximadamente 6 meses, desde la etapa de Iniciación hasta su lanzamiento.
\begin{itemize}
    \item Iniciación: del 31 de julio de 2025 al 31 de agosto de 2025.
    \item Pre-producción: del 1 de septiembre de 2025 al 6 de enero de 2025.
    \begin{itemize}
        \item Primer Vertical Slice: del 2 de diciembre de 2025 al 20 de diciciembre de 2025 
        \item Segundo Vertical Slice: del 20 de diciembre de 2025 al 6 de enero de 2026.
    \end{itemize}
    \item Producción y Lanzamiento: del 6 de enero de 2026 al 20 de enero de 2026.
\end{itemize}

\begin{figure}[H]
\centering
\includegraphics[width=\textwidth]{timeline_v2.png}
\caption{Línea del tiempo del proyecto. Creada por autor.}
\label{fig:x timeline}
\end{figure}
\subsection{Aspectos positivos}
\paragraph{Iniciación y participar en jams} Se encontró que participar en jams durante la etapa de iniciación fue altamente beneficioso para el proyecto. Es una excelente herramienta para probar ideas y validar si vale la pena expandirlas en un juego completo.
\paragraph{Playtesting durante la etapa de pre-producción} Se encontró que la herramienta de playtesting fue esencial para validar si el proyecto iba por buen camino. Mostrar el juego a otras personas ayudó a reducir la incertidumbre y a encontrar errores no detectados previamente.
\paragraph{Game Design Macro} El \textit{Game Design Macro} fue otra herramienta que resultó muy valiosa para el proyecto. Al establecer las tareas requeridas para finalizar el juego, ayudó a que la etapa de producción estuviera relativamente libre de dificultades.
%
%
\subsection{Aspectos a mejorar}
\paragraph{Problemas de alcance durante pre-producción} Una de las mayores dificultades que se encontraron en el proyecto fue establecer un alcance aceptable para el juego. Si bien es importante experimentar durante la etapa de pre-producción, se considera que también es necesario establecer límites que eviten estancar el proyecto. En base a la experiencia durante este desarrollo, se proponen 2 posibles soluciones a este problema:
\begin{itemize}
    \item \textbf{Fecha de finalización inalterable}: una opción es establecer una fecha donde la etapa debe sí o sí terminar. Esto implicaría detener todo el desarrollo de nuevas mecánicas o ideas, y enfocarse en generar contenido con los sistemas que ya están implementados.
    \item \textbf{Playtesting desde el principio}: Como se mencionó anteriormente, llevar a cabo rondas de playtesting ayudó enormemente a eliminar la incertidumbre respecto al proyecto. Es inevitable tener dudas respecto al juego, y cuestionarse si es divertido o interesante. Al principio del proyecto, se trató de minimizar este problema añadiendo mecánicas nuevas, que extendieron el tiempo de desarrollo innecesariamente. Se considera que si se hubiera mostrado el juego desde el principio del desarrollo, se hubiera reducido el tiempo dedicado a crear sistemas que luego no se utilizaron, y, en cambio, se hubiera utilizado ese tiempo para refinar los sistemas que eventualmente quedaron como parte final del juego.
\end{itemize}
\paragraph{Omisión de la etapa de post-producción} El problema anterior produjo una nueva dificultad; no quedó tiempo para realizar una realizar la etapa de post-producción. Esto significa que el contenido durante esta etapa no fue probado, y por lo tanto puede incluir bugs y niveles o mecánicas aburridas o desbalanceadas.
\paragraph{Uso limitado de metodologías de desarrollo} Debido a la naturaleza creativa de crear un videojuego, se encontraron varias dificultades para utilizar herramientas como metodologías ágiles o Kanban. Esto se debe a que las primeras etapas, Iniciación y Pre-Producción, son irregulares por defecto. El objetivo de estas fases es explorar ideas y probar mecánicas, lo que dificulta el uso de herramientas que tratan de establecer tareas específicas y medir el tiempo que toman. A esto se suma que muchas veces el primer intento de implementar el sistema no representa el tiempo real que toma una vez se conoce el proceso. Un ejemplo de esto fue la creación de niveles. Los primeros niveles tomaron considerablemente más tiempo que los últimos, donde la práctica ya había agilizado el desarrollo.