\par Una forma de observar el progreso de un proyecto es recolectando información sobre las tareas que realizan los developers, y observar si aparecen patrones que indiquen la eficiencia del trabajo realizado. A continuación, se nombran algunas de las métricas comúnmente estudiadas.
%
\subsubsection{Lead Time}
\par Registra el tiempo que toma una tarea, desde que se extrae del Product Backlog hasta que es finalmente entregado. Sirve para estimar tiempos de entrega, medido en días \cite{gaeteEnfoqueAplicacionAgil2021}.
%
\subsubsection{Touch Time}
Registra el tiempo real que toma finalizar una tarea, descontando los posibles periodos de inactividad. Se mide en horas de trabajo \cite{gaeteEnfoqueAplicacionAgil2021}.
%
\subsubsection{Velocidad}
Cantidad de ítems de trabajo completados en cada iteración. Mide la productividad del equipo, y busca maximizar el número de requisitos terminados \cite{gaeteEnfoqueAplicacionAgil2021}.
%
\subsubsection{Requisitos no completados}
Registra la cantidad de requisitos que no se completaron durante la iteración a pesar de haberse establecido como uno \cite{gaeteEnfoqueAplicacionAgil2021}.
