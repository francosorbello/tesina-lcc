\par A medida que la tecnología avanza, el software se vuelve cada vez más complejo, con programas que requieren múltiples funcionalidades, integración con otros sistemas y constante cambio para ajustarse a las necesidades de los clientes.
\par Esta complejidad plantea desafíos a la hora de crear y mantener un software. Para abordar esta problemática, surgieron las metodologías de desarrollo de software (SDLC en inglés). El objetivo principal de las SDLC es simplificar el proceso de desarrollo y mantenimiento de software. Esto se logra mediante el uso de frameworks que definen una estructura clara para cada fase del ciclo de vida del software, desde la planificación inicial y el diseño, hasta la implementación, las pruebas, el despliegue y el mantenimiento a largo plazo \cite{sStudySoftwareDevelopment2017,rupareliaSoftwareDevelopmentLifecycle2010}. Generalmente, las SDLC comparten una serie de fases comunes a todos los frameworks.

\begin{enumerate}
    \item \textbf{Planning}: En esta etapa se comunica con el cliente y se establece la viabilidad y costo de desarrollar el software \cite{sStudySoftwareDevelopment2017,dwivediComparativeStudyVarious2022} . El objetivo es evaluar los objetivos de los participantes y reunir los requerimientos que ayuden a definir el producto final \cite{pressmanIngenieriaSoftwareEnfoque2013}
    \item \textbf{Definición de requerimientos}: En este paso se ahonda en las necesidades del cliente, mediante conversaciones y entrevistas con los consumidores y dueños del software. Luego, el equipo analiza cada requerimiento y evalúa las posibles ventajas y desventajas de llevarlo a cabo, además de las posibles dificultades técnicas y de diseño que traiga \cite{pressmanIngenieriaSoftwareEnfoque2013,sStudySoftwareDevelopment2017,dwivediComparativeStudyVarious2022}.
    \item \textbf{Diseño}: Las especificaciones del paso anterior se utilizan para plantear el diseño del software. El objetivo es crear un plan que guíe el desarrollo \cite{pressmanIngenieriaSoftwareEnfoque2013,sStudySoftwareDevelopment2017,dwivediComparativeStudyVarious2022}. En ciertas metodologías, este plan es final, e indica los objetivos a cumplir, los tiempos en los que se llevarán a cabo, y las entregas del software.
    \item \textbf{Desarrollo}: Como el nombre lo indica, esta etapa consiste en la creación de código y assets requeridos para el proyecto \cite{pressmanIngenieriaSoftwareEnfoque2013,sStudySoftwareDevelopment2017,dwivediComparativeStudyVarious2022}. Algunas herramientas comunes son lenguajes de programación como C\# o Javascript, además de debuggers\footnote{debugger:programa utilizado para detectar errores en un software.} y otras herramientas de testing  \footnote{testing: proceso de detectar errores en un software.}.
    \item \textbf{Testing}: Se crea un ambiente de testing, donde los desarrolladores buscan errores en el software y sus distintas funcionalidades \cite{pressmanIngenieriaSoftwareEnfoque2013,sStudySoftwareDevelopment2017,dwivediComparativeStudyVarious2022}.
    \item \textbf{Deployment}: El software se vuelve accesible para los clientes. En algunos casos, se provee entrenamiento a los usuarios \cite{pressmanIngenieriaSoftwareEnfoque2013,sStudySoftwareDevelopment2017,dwivediComparativeStudyVarious2022}.
    \item \textbf{Mantenimiento}: Se continúa dando soporte al software a medida que es utilizado por los clientes \cite{pressmanIngenieriaSoftwareEnfoque2013,sStudySoftwareDevelopment2017,dwivediComparativeStudyVarious2022}.
\end{enumerate}
