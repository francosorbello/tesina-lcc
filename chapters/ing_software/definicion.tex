En la actualidad, el software es una institución transversal al funcionamiento del mundo. Los gobiernos y otras entidades públicas son manejados mediante computadoras. Tanto el sector público como el privado utilizan la tecnología para automatizar la mayoría de industrias. Inclusive el arte es afectado por el software, con programas que asisten a la creación de música, pintura o video, entre otros \cite{sommervilleIngenieriaSoftware9a2011}.
\bigbreak
Formalmente, un software es definido como un “conjunto de programas de cómputo, procedimientos, reglas, documentación y datos asociados, que forman parte de las operaciones de un sistema de computación” \cite{alarconaldanaMetodologiaParaDesarrollo2020}. Sin embargo, el software tiene una serie de características que dificultan una definición única.
\begin{enumerate}
    \item El software se desarrolla o modifica con intelecto: los programas son productos intangibles, y por lo tanto no están regidos por procesos de fabricación comúnes \cite{sommervilleIngenieriaSoftware9a2011,pressmanIngenieriaSoftwareEnfoque2013}
    \item El software no se desgasta: el software no es susceptible a las condiciones físicas que, por ejemplo, producen un desgaste en la usabilidad de un producto. Sin embargo, el software sí se deteriora. A lo largo de su desarrollo, el software sufre cambios que producen fallas. Estos cambios suelen traer mayor complejidad, por lo que estos errores aumenten en cantidad y dificultad \cite{sommervilleIngenieriaSoftware9a2011,pressmanIngenieriaSoftwareEnfoque2013}.
    \item El software se construye modularmente: a medida que una disciplina de la ingeniería avanza, se crean componentes típicos para utilizar en el diseño. Por ejemplo, tornillos, o circuitos integrados que están pre construidos. Algo similar sucede en el software, donde se crean componentes reutilizables en otros programas, y sirven como ladrillos para construir nuevos programas \cite{sommervilleIngenieriaSoftware9a2011,pressmanIngenieriaSoftwareEnfoque2013}.
\end{enumerate}
\bigbreak
Por otro lado, a lo largo de los años han aparecido distintos tipos de software que satisfacen mercados variados. Se pueden encontrar:
\begin{itemize}
    \item \textbf{Software de sistemas}, que dan servicios a otros programas. Algunos ejemplos son compiladores, software de redes, o sistemas operativos \cite{pressmanIngenieriaSoftwareEnfoque2013}.
    \item \textbf{Aplicaciones independientes}, programas aislados que resuelven una necesidad específica y corren localmente en una PC \cite{sommervilleIngenieriaSoftware9a2011,pressmanIngenieriaSoftwareEnfoque2013}.
    \item \textbf{Software de ingeniería y ciencias}, que ayudan en procesos de investigación, simulación o cálculos matemáticos complejos.
    \item \textbf{Software embebido}, pensado para controlar dispositivos de hardware pequeños. Un ejemplo sería el panel de un microondas.
    \item \textbf{Aplicaciones web}, distinguidas por su fácil acceso, por lo general desde un navegador web.
    \item \textbf{Software de inteligencia artificial}, que utilizan algoritmos complejos para analizar grandes cantidades de datos.
\end{itemize}
\par
Esta tesina se enfoca en el videojuego, un arte nacido específicamente del software, y que no es capaz de existir fuera de él. Un aspecto interesante de ellos es que se ajustan a varios modelos. Pueden ser programas aislados, o tener conexión a servidores. En algunos casos, se pueden jugar desde un navegador web \cite{PhaserFastFun}. Inclusive, algunos motores gráficos se utilizan para simulación y entrenamiento de inteligencia artificial \cite{MLAgentsOverviewML,UnrealEngineAdvanced}.