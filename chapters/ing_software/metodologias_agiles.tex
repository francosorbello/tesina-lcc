\par Las metodologías ágiles aparecen por la necesidad de una metodología flexible y capaz de adaptarse a los constantes cambios \cite{bartoszScrumVideoGames2023}. Estas metodologías se desarrollan en los años 90, y nacen de la frustración que las empresas percibían respecto a metodologías lineales \cite{sommervilleIngenieriaSoftware9a2011,pressmanIngenieriaSoftwareEnfoque2013}. 
El objetivo es que el equipo de desarrollo se enfoque en crear el software en vez de diseñar y documentar. Además, toman conceptos del desarrollo incremental, donde el producto se avanza generando iteraciones que son entregables a un cliente \cite{pressmanIngenieriaSoftwareEnfoque2013}.

\par Hay varias implementaciones de metodologías ágiles. Una de las más populares es Scrum, explicada en la sección a continuación.